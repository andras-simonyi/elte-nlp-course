\documentclass[style=upen, size=14pt]{powerdot}
\definecolor{arany}{RGB}{255,242,0}
\hypersetup{backref=page}
\hypersetup{
    colorlinks=true,
    linkcolor=cyan,
    filecolor=magenta,      
    urlcolor=cyan}
\usepackage{graphicx}
\usepackage{amsmath}
\DeclareMathOperator*{\argmax}{argmax}
\DeclareMathOperator*{\argmin}{argmin}
\usepackage{amssymb}
\usepackage{stmaryrd}
\usepackage[latin2]{inputenc}
%\usepackage[magyar]{babel}
%\usepackage{euler}
\usepackage{tikz}
\usepackage{tikz-qtree}
\usepackage{tikz-dependency}
\usepackage{linguex}
\tikzset{every tree node/.style={align=center,anchor=north}}
%\usepackage{tabularx}
%\usepackage{threeparttable}
%\usepackage{color}
%\selectlanguage{english}
%\frenchspacing
\newcommand{\nd}{\noindent}
\newcommand{\Val}{\mathop{\mathit{Val}}}
\newcommand{\gold}{\color{arany}}
%\usepackage{tikz}
%\usepackage{tikz-qtree}
\newcommand{\qed}{\hfill\mbox{\raggedright \rule{.1in}{.1in}}}
\def\es{\mathbin\land}
\newtheorem{defi}{Definition}
\newtheorem{axioma}{Axiom}
\newtheorem{tetel}{Theorem}
\newtheorem{prop}{Proposition}
\newtheorem{lemma}{Lemma}
\begin{document}

\title{Natural Language Processing\\~~\\Lecture 3\\Tokenization}
% \author{}

\date{2021}
\maketitle


\begin{slide}{Representation levels}
  Natural languages are very complex sign systems, and their signs (words,
  phrases, sentences etc.) have dramatically more internal structure than
  ordinary symbols. Linguists typically distinguish at least the following four
  levels of linguistic representation in a linguistic sign:\footnote{The
    discussion of representation levels and grammars in this section closely
    follows Marcus Kracht's
    \href{https://linguistics.ucla.edu/people/Kracht/courses/ling20-fall07/ling-intro.pdf}{Intoduction to Linguistics}, which is highly recommended.}
  \begin{itemize}
  \item {\gold phonological structure}: the level of individual sounds, or, in
    written language, written symbols, letters;
  \item {\gold morphological structure}: the level of \emph{morphemes}, i.e.,
    minimal  meaningful linguistic units, and their organization into \emph{words};
  \end{itemize}
\end{slide}


\end{document}



%%% Local Variables:
%%% mode: latex
%%% TeX-master: t
%%% End:

% LocalWords:  Tokenization
