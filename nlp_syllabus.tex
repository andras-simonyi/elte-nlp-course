% Created 2020-09-15 Tue 10:59
% Intended LaTeX compiler: pdflatex
\documentclass[10pt, a4paper]{article}
\usepackage{times}
\usepackage[utf8]{inputenc}
\usepackage[T1]{fontenc}
\usepackage{graphicx}
\usepackage{grffile}
\usepackage{longtable}
\usepackage{wrapfig}
\usepackage{rotating}
\usepackage[normalem]{ulem}
\usepackage{amsmath}
\usepackage{textcomp}
\usepackage{amssymb}
\usepackage{capt-of}
\usepackage{hyperref}
\author{András Simonyi}
\date{2021}
\title{Natural Language Processing}
\hypersetup{
 pdfauthor={András Simonyi},
 pdftitle={Course plan},
 pdfkeywords={},
 pdfsubject={},
 pdfcreator={Emacs 27.1 (Org mode 9.3.7)}, 
 pdflang={English}}
\begin{document}

\maketitle
\section*{Lectures}
\subsection*{Introduction}
\begin{enumerate}
\item Introduction: what is NLP, related fields, typical applications, central
  themes
\item Overview of Computational Linguistics and the components of the
  traditional NLP pipeline: tokenization and sentence segmentation, POS tagging
  and morphological analysis, syntax, semantics, pragmatics
\end{enumerate}
\subsection*{Basic methods}
\begin{enumerate}
\item Tokenization: regexes, normalization and edit distance, subword
  tokenization methods
\item N-gram based language modeling
\item Classification and sequence tagging: Sentiment analysis,
POS-tagging, morphology, NER
\item Dependency parsing
\item Lexical semantics: WSD and related databases (WordNet,
ConceptNet, Wikipedia), LSA
\end{enumerate}
\subsection*{Neural methods}
\begin{enumerate}
\item Word2vec and neural embeddings
\item Recurrent Neural Networks and RNN-based Language Models
\item Machine Translation, Seq2Seq and Attention
\item Contextual word representations and fine tuning: BERT and co.
\end{enumerate}
\end{document}
%%% Local Variables:
%%% mode: latex
%%% TeX-master: t
%%% End:
